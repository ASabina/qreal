\documentclass[a5paper]{article}
\usepackage[a5paper, top=17mm, bottom=17mm, left=17mm, right=17mm]{geometry}
\usepackage[utf8]{inputenc}
\usepackage[T2A,T1]{fontenc}
\usepackage[colorlinks,filecolor=blue,citecolor=green,unicode,pdftex]{hyperref}
\usepackage{cmap}
\usepackage[english,russian]{babel}
\usepackage{amsmath}
\usepackage{amssymb,amsfonts,textcomp}
\usepackage{color}
\usepackage{array}
\usepackage{hhline}
\hypersetup{colorlinks=true, linkcolor=blue, citecolor=blue, filecolor=blue, urlcolor=blue, pdftitle=1, pdfauthor=, pdfsubject=, pdfkeywords=}
% \usepackage[pdftex]{graphicx}
\usepackage{graphicx}
\usepackage{epigraph}
% Раскомментировать тем, у кого этот пакет есть. Шрифт станет заметно красивее.
%\usepackage{literat}
\usepackage{indentfirst}

\sloppy
\pagestyle{plain}
%\pagestyle{empty}

\title{Мышиные жесты, ПЫЩЬ}

\author{Т.А. Брыксин \and Ю.В. Литвинов \and М.С. Осечкина}
\date{}
\begin{document}

\maketitle
\thispagestyle{empty}

\epigraph{Цивилизация движется вперед путем увеличения числа операций, которые мы можем осуществлять, не раздумывая над ними}%
         {Альфред Норт Уайтхед}

\begin{quote}
\small\noindent
Аннотация, чо
\end{quote}

\section*{Введение}
Одной из особенностей разработки, управляемой моделями (model-driven development, MDD), является активное использование 
визуальных языков. Практически все действия, выполняемые в CASE-средствах или других используемых инструментах так или 
иначе сводятся к манипуляциям над элементами этих языков и связями между ними. 

Эффективность любого используемого инструмента определяется тем, насколько удобно и быстро он позволяет выполнять те операции, 
для которых этот инструмент предназначен. В процессе разработки моделей одними из наиболее часто выполняемых действий над объектами 
на диаграммах являются их создание и удаление.  В большинстве CASE-средств для того, чтобы создать нужный объект на диаграмме, 
необходимо найти его либо на панели инструментов, либо выбрать в меню, а затем указать место на диаграмме, где бы мы хотели этот 
элемент разместить. В большинстве инструментариев также возможен вариант создания объектов «перетаскиванием» (drag and drop) их из 
палитры элементов соответствующей диаграммы. То есть даже для такой базовой операции, как создание нового элемента, разработчику нужно 
совершить не только набор чисто механических действий, но еще и, скажем, вспомнить, на какой вкладке палитры или в каком меню находится 
нужный ему элемент, тем самым переключая контекст с продумывания иерархии создаваемых моделей на особенности манипуляции 
используемым инструментом. Нам кажется, что данную операцию можно и нужно автоматизировать, причем сделать это нужно максимально удобным для 
пользователей CASE-средств. В данной статье в качестве такого решения рассматривается подход, основанный на жестах мышью. 
Предлагается с каждым элементом ассоциировать определенный жест мышью, выполненный с каким-либо модификатором (скажем, с зажатой 
правой кнопкой мыши или клавишей Alt) и при выполнении этого жеста создавать в данном месте соответствующий объект. 

\section{Постановка задачи}
В качестве CASE-пакета, в котором было решено внедрить и опробовать данный подход, была выбрана система QReal, разрабатываемая на кафедре 
системного программирования СПбГУ. 

Одной из важных особенностей архитектуры QReal является возможность расширения набора графических редакторов. CASE-система состоит из 
некоего ``ядра'', реализующего абстрактную фунциональность редакторов и элементов, создаваемых на диаграммах, и всей необходимой специфики 
конкретных диаграмм, генерящейся из метамоделей соответствующих редакторов (см. рис. 1). Более подробно об архитектуре QReal и предлагаемом 
подходе к  созданию новых визуальных рекдакторов можно прочитать в [1488]. 

[Рис. 1. красивая картинка с архитектурой]

Для данного исследования потенциальная расширяемость QReal означает то, что и предлагаемое решение по встраиванию в среду распознавания
и выполнения команд, соответствующих тем или иным жестам, также не должно зависеть от текущего набора графических редакторов. Так, например,
мы должны иметь возможность задания жеста и ассоциирования его с определенным элементом еще в метамодели, к тому же все необходимые 
для распознавания данные должны автоматически создаваться при генерации исходного кода плагина по метамодели редактора и в идеале не требовать 
никаких дополнительных действий от пользователя. 

\section{Обзор использования жестов в других приложениях}

игры, браузеры, Visual Paradigm

\section{Обзор и выбор алгоритмов распознавания}

\section{Реализация}

\section{Направления дальнейшего исследования}

\end{document}






