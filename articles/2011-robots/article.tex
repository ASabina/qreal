\documentclass[a5paper]{article}
\usepackage[a5paper, top=17mm, bottom=17mm, left=17mm, right=17mm]{geometry}
\usepackage[utf8]{inputenc}
\usepackage[T2A,T1]{fontenc}
\usepackage[colorlinks,filecolor=blue,citecolor=green,unicode,pdftex]{hyperref}
\usepackage{cmap}
\usepackage[english,russian]{babel}
\usepackage{amsmath}
\usepackage{amssymb,amsfonts,textcomp}
\usepackage{color}
\usepackage{array}
\usepackage{hhline}
\hypersetup{colorlinks=true, linkcolor=blue, citecolor=blue, filecolor=blue, urlcolor=blue, pdftitle=1, pdfauthor=, pdfsubject=, pdfkeywords=}
% \usepackage[pdftex]{graphicx}
\usepackage{graphicx}
% \usepackage{epigraph}
% Раскомментировать тем, у кого этот пакет есть. Шрифт станет заметно красивее.
%\usepackage{literat}
\usepackage{indentfirst}
\usepackage{wrapfig}

\sloppy
\pagestyle{plain}

\title{QReal:Robots. Графическая среда программирования роботов.}

\author{А.С. Кузенкова \and А.О. Дерипаска \and А.И. Птахина}
\date{}
\begin{document}

\maketitle
\thispagestyle{empty}

\begin{quote}
\small\noindent
Краткий абстракт, пара предложений о чём статья.
\end{quote}

\section*{Введение} (2/3 - 1 стр)
\begin{enumerate}
  \item О том, как хорошо использовать роботов в школьном образовании
  \begin{enumerate}
    \item Сослаться на Ершова, про LOGO
    \item По мотивам \begin{verbatim} https://docs.google.com/document/d/1ODzZQf_8NSRdAMJIlh_xYiAIPRMpSnCKz6ZBDv0Ehpg/edit?hl=en&authkey=CMGp2tYE# \end{verbatim}
  \end{enumerate}
  \item О том, как хорошо использовать средства визуального программирования для роботов. Опять же, по мотивам ссылки выше.
  \item Однако же, просто так визуальный редактор не сделать, поэтому надо использовать Технологию (тут по мотивам докладов про метаредактор, например, \begin{verbatim} https://github.com/qreal/qreal/blob/master/articles/2011-metamodelling/article.tex \end{verbatim} )
\end{enumerate}

\section{Средства метамоделирования QReal} (1 стр)
\begin{enumerate}
  \item Небольшой рассказ о QReal вообще ("На кафедре системного программирования СПбГУ в течение нескольких лет" и т.д.)
  \item Метаредактор ("XML-описания редакторов можно генерировать по визуальным моделям" и в таком духе, без подробностей)
  \item Редактор шейпов (в духе текста из политеховских тезисов, тоже без подробностей)
\end{enumerate}

\section{Язык программирования роботов} (2 стр)
\begin{enumerate}
  \item Возникла задача создания средства визуального программирования роботов, которая нас очень заинтересовала, поскольку это хороший пример визуального предметно-ориентированного языка
  \item Робот управляется по Bluetooth, программа представляется в виде последовательности блоков, какие блоки бывают, и всё такое. По мотивам \begin{verbatim} https://docs.google.com/document/d/1KkeTAJlu1zUeLv1_kdwXxkoKwIqS6VGb1OehdCLZDoY/edit?hl=en&authkey=CI6Zre4N# \end{verbatim}
  \item А ещё у нас есть математические выражения, и тут немного подробнее, что можно писать и как оно работает (например, про переменные доступа к сенсорам, что среда опрашивает робот раз в N миллисекунд)
  \item А ещё у нас есть 2д-модель робота. Тут поподробнее зачем она нужна, что умеет делать, и что хочется научить её делать в будущем.
  \item И что в итоге получилось: кроссплатформенная опенсорсная среда, вполне годная ля школы (опять же, по мотивам доклада в 239, только не очень рекламно)
  \item Картинка с примером диаграммы
\end{enumerate}

\section*{Заключение} (1/3 стр)
Сделали, опробовали, доложились на конференциях, вызвали интерес, сейчас хотим внедрять в школы. (1 абзац)

\begin{thebibliography}{9001}
  \bibitem{qReal} А.Н. Терехов, Т.А. Брыксин, Ю.В. Литвинов и др., Архитектура среды визуального моделирования QReal. // Системное программирование. Вып. 4. СПб.: Изд-во СПбГУ. 2009, С. 171-196
  \bibitem{theBook} Kelly, S., Tolvanen, J. Domain-Specific Modeling: Enabling Full Code Generation // Wiley-IEEE Computer Society Press. 2008. 448 pp.
\end{thebibliography}

\end{document}